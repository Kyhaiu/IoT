\documentclass[12pt,a4paper]{article}
\usepackage[brazil]{babel}
\usepackage[utf8]{inputenc}
\usepackage[T1]{fontenc}
\usepackage{geometry}
\usepackage{setspace}
\usepackage{indentfirst}
\usepackage{titlesec}
\usepackage{enumitem}
\usepackage{graphicx} % Para inserir imagens
\usepackage{listings} % Para inserir código
\usepackage{xcolor} % Para colorir código
\usepackage{hyperref} % Para links

% =============================================
% CONFIGURAÇÕES - NÃO ALTERAR
\geometry{a4paper, margin=2.5cm}
\setstretch{1.3}
\titleformat{\section}{\large\bfseries}{\thesection}{1em}{}
\setlist[itemize]{noitemsep, topsep=0pt}

% Configuração para código
\lstset{
    basicstyle=\ttfamily\small,
    commentstyle=\color{green!40!black},
    keywordstyle=\color{blue}\bfseries,
    frame=single,
    breaklines=true,
    numbers=left,
    numberstyle=\tiny\color{gray},
    showstringspaces=false,
    language=C++,
}

% Configuração de links
\hypersetup{
    colorlinks=true,
    linkcolor=blue,
    filecolor=magenta,      
    urlcolor=blue,
}
% =============================================

% =============================================
% PREENCHA ESTA SEÇÃO COM OS DADOS DO SEU GRUPO
% =============================================
\newcommand{\tituloDoProjeto}{[TÍTULO DO SEU PROJETO AQUI]}
\newcommand{\nomeAlunoA}{[Nome Completo do Aluno A]}
\newcommand{\nomeAlunoB}{[Nome Completo do Aluno B]}
\newcommand{\nomeAlunoC}{[Nome Completo do Aluno C]}
\newcommand{\nomeProfessor}{[Nome do Professor]}
\newcommand{\nomeDisciplina}{[Nome da Disciplina]}
\newcommand{\dataEntrega}{[DD/MM/AAAA]}
% =============================================

\begin{document}

% =============================================
% CAPA - PREENCHA OS CAMPOS ACIMA
% =============================================
\begin{titlepage}
    \centering
    \vspace*{2cm}
    
    {\huge\bfseries \tituloDoProjeto\par}
    
    \vspace{2cm}
    {\Large Trabalho Prático de Internet das Coisas\par}
    
    \vspace{1.5cm}
    {\large
    \begin{tabular}{c}
        \nomeAlunoA \\
        \nomeAlunoB \\
        \nomeAlunoC
    \end{tabular}\par}
    
    \vfill
    
    {\large Professor: \nomeProfessor\par}
    {\large \nomeDisciplina\par}
    {\large Data de Entrega: \dataEntrega\par}
    
    \vspace{2cm}
\end{titlepage}


% =============================================
% SUMÁRIO (OPCIONAL - REMOVA SE NÃO QUISER)
% =============================================
\tableofcontents
\newpage


% =============================================
% SEÇÃO 1: INTRODUÇÃO 
% INSTRUÇÕES: Explique o contexto e motivação do projeto
% =============================================
\section{Introdução (OBRIGATÓRIO)}

% EXEMPLO:
% O sistema proposto visa monitorar a qualidade do ar em ambientes internos
% de escolas, alertando sobre excesso de CO2 e umidade inadequada.

[ESCREVA AQUI A INTRODUÇÃO DO SEU PROJETO. Explique:
- Qual problema você está resolvendo?
- Por que este problema é importante?
- Qual a solução proposta?]


% =============================================
% SEÇÃO 2: OBJETIVOS
% INSTRUÇÕES: Liste o que você pretende alcançar
% =============================================
\section{Objetivos (OBRIGATÓRIO)}

\subsection{Objetivo Geral}
[Defina aqui o objetivo principal. Ex: "Desenvolver um sistema de monitoramento de temperatura e umidade para estufas agrícolas com alertas automatizados."]

\subsection{Objetivos Específicos}
\begin{itemize}
    \item Descreva o objetivo específico 1. Ex: "Coletar dados de temperatura e umidade a cada 5 minutos."
    \item Descreva o objetivo específico 2. Ex: "Implementar comunicação Wi-Fi para envio de dados."
    \item Descreva o objetivo específico 3. Ex: "Criar dashboard web para visualização em tempo real."
    \item Adicione mais itens conforme necessário.
\end{itemize}


% =============================================
% SEÇÃO 3: REQUISITOS TÉCNICOS
% INSTRUÇÕES: Liste todos os componentes utilizados
% =============================================
\section{Requisitos Técnicos (OBRIGATÓRIO)}

INSTRUÇÕES: Liste todos os componentes utilizados

\subsection{Hardware}
\begin{itemize}
    \item \textbf{Microcontrolador}: [Ex: ESP32 DevKit V1]
    \item \textbf{Sensores}: [Ex: DHT22 (temperatura/umidade), MQ-135 (qualidade do ar)]
    \item \textbf{Atuadores}: [Ex: LED RGB, Buzzer (alertas sonoros)]
    \item \textbf{Componentes}: [Ex: Protoboard 400 pontos, Resistores 330Ω, Jumpers, Fonte 5V]
\end{itemize}

\subsection{Software}
\begin{itemize}
    \item \textbf{IDE}: [Ex: Arduino IDE 2.0]
    \item \textbf{Linguagem}: [Ex: C++]
    \item \textbf{Protocolos}: [Ex: MQTT via Wi-Fi]
    \item \textbf{Plataforma}: [Ex: Adafruit IO + ThingSpeak]
    \item \textbf{Bibliotecas}: [Ex: DHT.h, PubSubClient.h, WiFi.h]
\end{itemize}


% =============================================
% SEÇÃO 4: ARQUITETURA DO SISTEMA 
% INSTRUÇÕES: Descreva o fluxo de dados e inclua diagramas
% =============================================
\section{Arquitetura do Sistema (OPCIONAL)}

% INSTRUÇÃO: Insira um diagrama de blocos aqui
% Para adicionar imagem: descomente e ajuste o caminho
% \begin{figure}[h]
% \centering
% \includegraphics[width=0.8\textwidth]{caminho/para/seu/diagrama.png}
% \caption{Diagrama de Blocos do Sistema}
% \label{fig:diagrama}
% \end{figure}

\textbf{Descrição do Fluxo}:
[Ex: "Os sensores DHT22 e MQ-135 coletam dados que são lidos pelo ESP32 a cada 5 minutos. O microcontrolador processa os valores e publica via MQTT no tópico 'sala01/dados'. O Adafruit IO recebe os dados e exibe em dashboard, além de enviar alertas por e-mail quando os valores ultrapassam limites."]

\textbf{Arquitetura da Rede}:
[Descreva como os dispositivos se comunicam. Inclua IP, portas, topologia, etc.]


% =============================================
% SEÇÃO 5: DESENVOLVIMENTO
% INSTRUÇÕES: Detalhe a implementação
% =============================================
\section{Desenvolvimento (OBRIGATÓRIO)}

\subsection{Esquemático do Circuito}
% INSTRUÇÃO: Insira o esquemático elétrico aqui
% Use fotos, screenshots ou diagramas feitos com Fritzing, Tinkercad, etc.
% \begin{figure}[h]
% \centering
% \includegraphics[width=\textwidth]{caminho/para/esquematico.png}
% \caption{Esquemático do Circuito}
% \label{fig:esquematico}
% \end{figure}

[Liste as conexões dos pinos. Ex: "DHT22: VCC → 3.3V, GND → GND, DATA → GPIO 4"]

\subsection{Código Fonte (OBRIGATÓRIO)}
% INSTRUÇÃO: Insira trechos importantes do código
% Para código curto (< 1 página), cole diretamente:
\begin{lstlisting}[caption=Exemplo de leitura de sensor]
// [COLE SEU CÓDIGO AQUI COMENTADO]
// Exemplo:
#include <WiFi.h>
#include <PubSubClient.h>

void setup() {
  Serial.begin(115200);
  // Inicialize seus componentes
}
\end{lstlisting}

% Para código extenso, coloque apenas os trechos mais importantes
% e faça referência ao arquivo completo no GitHub

\textbf{Repositório com Código Completo}: \url{[LINK DO SEU REPOSITÓRIO NO GITHUB]}


% =============================================
% SEÇÃO 5: TESTES E RESULTADOS
% INSTRUÇÕES: Mostre que o sistema funciona
% =============================================
\section{Testes e Resultados (OBRIGATÓRIO)}

\subsection{Procedimentos de Teste}
[Descreva como testou. Ex: "Deixei o sistema operando por 72h, coletando dados a cada 5 min. Testes de estresse foram feitos expondo sensores a condições extremas."]

\subsection{Resultados Obtidos}
% INSTRUÇÃO: Inclua tabelas ou gráficos dos dados
% \begin{table}[h]
% \centering
% \caption{Dados coletados durante 24h}
% \begin{tabular}{|c|c|c|c|}
% \hline
% Horário & Temperatura (°C) & Umidade (\%) & CO2 (ppm) \\
% \hline
% 08:00 & 23.5 & 65 & 450 \\
% \hline
% \end{tabular}
% \end{table}

[Analise os resultados. Ex: "O sistema manteve 98\% de uptime. Latência média de envio: 2.3s."]

\subsection{Problemas Encontrados e Soluções}
\begin{itemize}
    \item \textbf{Problema}: [Ex: "DHT22 retornava valores erráticos."]
    \item \textbf{Solução}: [Ex: "Adicionado capacitor de 100nF entre VCC e GND."]
\end{itemize}


% =============================================
% POR FAVOR, LEIA AS DICAS ABAIXO ANTES DE ENTREGAR:
% =============================================

% DICAS IMPORTANTES:
% 1. REMOVA todos os textos em [MAIÚSCULAS] e coloque seu conteúdo
% 2. REMOVA os comentários de instrução (linhas que começam com %)
% 3. VERIFIQUE se todas as imagens estão no caminho correto
% 4. TESTE a compilação antes da entrega final
% 5. NOMEIE o arquivo: TRABALHO_IOT_GRUPO_X_NOMES.pdf

\end{document}